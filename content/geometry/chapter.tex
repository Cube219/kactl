\chapter{Geometry}

\section{Geometric primitives}
	\kactlimport{Point.h}
	\kactlimport{lineDistance.h}
	\kactlimport{SegmentDistance.h}
	\kactlimport{SegmentIntersection.h}
	\kactlimport{lineIntersection.h}
	\kactlimport{sideOf.h}
	\kactlimport{OnSegment.h}
	\kactlimport{linearTransformation.h}
	% \kactlimport{LineProjectionReflection.h}
	\kactlimport{Angle.h}

\section{Circles}
	\kactlimport{CircleIntersection.h}
	\kactlimport{CircleTangents.h}
	% \kactlimport{CircleLine.h}
	\kactlimport{CirclePolygonIntersection.h}
	\kactlimport{circumcircle.h}
	\kactlimport{MinimumEnclosingCircle.h}

\section{Polygons}
	\kactlimport{InsidePolygon.h}
	\kactlimport{PolygonArea.h}
	\kactlimport{PolygonCenter.h}
	\kactlimport{PolygonCut.h}
	% \kactlimport{PolygonUnion.h}
	\kactlimport{ConvexHull.h}
	\kactlimport{HullDiameter.h}
	\kactlimport{PointInsideHull.h}
	\kactlimport{LineHullIntersection.h}

\section{Misc. Point Set Problems}
	\kactlimport{ClosestPair.h}
	% \kactlimport{ManhattanMST.h}
	\kactlimport{kdTree.h}
	% \kactlimport{DelaunayTriangulation.h}
	\kactlimport{FastDelaunay.h}

	\subsection{픽의 정리}
		모든 꼭짓점이 격자점 위에 존재하는 단순 다각형의 넓이를 A, 격자 다각형 내부에 있는 격자점의 수를 i, 변 위에 있는 격자점의 수를 b라고 하면\\
		$A=i+\frac{b}{2}-1$

	\subsection{평면 그래프}
		꼭짓점의 수를 v, 변의 수를 e, 면의 수를 f라고 하면, 평면 그래프의 경우\\
		$v-e+f=2$

\section{3D}
	\kactlimport{PolyhedronVolume.h}
	\kactlimport{Point3D.h}
	\kactlimport{3dHull.h}
	\kactlimport{sphericalDistance.h}
